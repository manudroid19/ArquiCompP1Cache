%%%%%%%%%%%%%%%%%%%%%%%%%%%%%%%%%%%%%%%%%%%%%%%%%%%%%%%%%%%%%%%%%%%%%%%%%%%%%%%%
%2345678901234567890123456789012345678901234567890123456789012345678901234567890
%        1         2         3         4         5         6         7         8

\documentclass[letterpaper, 10 pt,spanish, conference]{ieeeconf}  % Comment this line out
\usepackage[spanish]{babel}
\selectlanguage{spanish}
\usepackage[utf8]{inputenc}
\usepackage{hyperref, mathtools}

\usepackage{amsmath, amsthm, amssymb, amsfonts, amscd}
\DeclarePairedDelimiter{\ceil}{\lceil}{\rceil}
                                          % if you need a4paper
%\documentclass[a4paper, 10pt, conference]{ieeeconf}      % Use this line for a4
                                                          % paper

\IEEEoverridecommandlockouts                              % This command is only
                                                          % needed if you want to
                                                          % use the \thanks command
\overrideIEEEmargins
% See the \addtolength command later in the file to balance the column lengths
% on the last page of the document



% The following packages can be found on http:\\www.ctan.org
%\usepackage{graphics} % for pdf, bitmapped graphics files
%\usepackage{epsfig} % for postscript graphics files
%\usepackage{mathptmx} % assumes new font selection scheme installed
%\usepackage{times} % assumes new font selection scheme installed
%\usepackage{amsmath} % assumes amsmath package installed
%\usepackage{amssymb}  % assumes amsmath package installed

\title{\LARGE \bf
Xerarquía de Memoria Caché: Estudo do Efecto da Localidade das Referencias a Memoria nas Prestacións dos Programas en Microprocesadores
}

%\author{ \parbox{3 in}{\centering Huibert Kwakernaak*
%         \thanks{*Use the $\backslash$thanks command to put information here}\\
%         Faculty of Electrical Engineering, Mathematics and Computer Science\\
%         University of Twente\\
%         7500 AE Enschede, The Netherlands\\
%         {\tt\small h.kwakernaak@autsubmit.com}}
%         \hspace*{ 0.5 in}
%         \parbox{3 in}{ \centering Pradeep Misra**
%         \thanks{**The footnote marks may be inserted manually}\\
%        Department of Electrical Engineering \\
%         Wright State University\\
%         Dayton, OH 45435, USA\\
%         {\tt\small pmisra@cs.wright.edu}}
%}

\author{Uxío García e Manuel de Prada}


\begin{document}



\maketitle
\thispagestyle{empty}
\pagestyle{empty}


%%%%%%%%%%%%%%%%%%%%%%%%%%%%%%%%%%%%%%%%%%%%%%%%%%%%%%%%%%%%%%%%%%%%%%%%%%%%%%%%
\begin{abstract}

Este documento pretende ilustrar a importancia e o impacto sobre o rendemento do prefetching e a localidade temporal e espacial nas referencias a memoria a través de distintos patróns de acceso e conxuntos de datos. Para poder xustificar devandita repercusión, estudase o custe temporal obtido en resultados de experimentos independentes sobre unha implementación en C dun programa que contén un bucle de computación, variando distintos parámetros de execución. 

\end{abstract}


%%%%%%%%%%%%%%%%%%%%%%%%%%%%%%%%%%%%%%%%%%%%%%%%%%%%%%%%%%%%%%%%%%%%%%%%%%%%%%%%
\section{INTRODUCIÓN}

A optimización do rendemento dos programas informáticos é un dos principais obxectivos de científicos e enxeñeiros dende a aparición da computación. Durante estas primeiras etapas de investigación xurdiu a hipótese de que o principio de Pareto podía ser aplicado a este campo. Esta hipótese tomaba a seguinte forma: "Un programa dedica o 90\% do seu tempo de execución no 10\% do código". A pesar de ser unha afirmación que nun principio non estaba apoidada por unha teoría sólida, comprobouse que era bastante certa experimentalmente. Isto favoreceu a aparición dun novo concepto, coñecido hoxe en día como principio de localidade, que describe moi ben o comportamento esperado do software medio e permite mellorar o rendemento.

O principio de localidade, tamén coñecido como localidade de referencia, é o termo empregado para describir a tendencia dun procesador (polo tanto, dos códigos) a acceder frecuentemente ó mesmo conxunto de direccións de memoria de forma repetida durante un curto período de tempo. Dentro do fenómeno da localidade, podemos distinguir dous casos: a localidade espacial e a temporal. A primeira refírese ao uso de datos cuxas direccións de memoria están relacionadas ou próximas, mentres que a segunda, como o seu propio nome indica, alude ao uso repetido dun mesmo conxunto de datos nun breve período de tempo.

Ademais, tamén ten un forte impacto no rendemento o prefetching que realiza o procesador automaticamente por hardware. Foi introducido xa dende os primeiros microprocesadores para mellorar o cacheado baseándose na localidade espacial dos datos e das instrucións. Partindo do concepto que empregan xa as memorias caché ao cargar nunha liña non só a posición que se pide senón tamén as veciñas, o prefetching consiste en predicir segundo o patrón de acceso cales serán as seguintes direccións solicitadas e precargalas antes de que se produza o fallo caché. Isto é máis sinxelo nas instrucións pero pode requirir unha análise complexa sobre os datos, para ser eficaz.

\section{Descrición do Experimento}

Como xa se apuntou previamente, o experimento realizado baseouse en realizar repetivamente unha operación de reducción sobre un vector dinámico e medir o número de ciclos medios por acceso a memoria dos elementos do vector, variando certos parámetros e observando o impacto no rendemento. Para seguir a exposición dun xeito máis claro, introducimos a seguinte notación:

\begin{itemize}
    \item A: Vector dinámico que almacena os elementos (de tipo double) que son sumados na reducción.
    \item S: Vector de N elementos de almacena a suma acumulada que computamos, repetida N veces a efectos estadísticos.
    \item D: Parámetro que describirá a dispersión dos accesos a memoria.
    \item L: Parámetro que depende do tamaño das cachés e que describirá o número de liñas caché distintas a acceder en cada execución.
    \item R: Parámetro que describirá o número de accesos a memoria necesarios para tocar L liñas caché dado unha dispersión D determinada.
\end{itemize}

    En primeiro lugar, para poder iniciar o experimento,
    debemos determinar cales serán os parámetros a utilizar. Este experimento deberase realizar 35 veces, xa que se escollerán 5 valores distintos de D e 7 valores distintos de L. 
    
    Para poder identificar a influencia dos valores de D na eficiencia do programa, é interesante seleccionar valores de D distanciados entre si. Como partimos da restrición inicial de que este valor ten que estar entre 1 e 100, e tamén sabemos que non se deben seleccionar potencias de 2 por mor do tipo de caché coa que conta o equipo sobre o que se executaron as probas, os valores seleccionados foron 1, 5, 12, 51 e 93.
    
    Por outra parte, para definir os valores de L foi necesario consultar os tamaños das cachés L1 e L2. Na aula dispoñemos de procesadores i3-3240\cite{c1}, cuxas cachés L1 e L2 son de tamaño 32KB e 256KB, respectivamente. Tendo en conta que unha liña caché son 64 bytes, obtemos uns tamaños en liñas S1 e S2 de 512 e 4096 liñas caché, respectivamente. A partir deste resultado e dunhas constantes previamente especificadas, obtivemos os 7 valores de L: 256, 768, 2048, 3072, 8192, 16384 e 32768.
    
    Unha vez coñecidos D e L, xa se dispoñía de todo o necesario para definir o valor de R. Como xa se mencionou previamente, este valor correspóndese co necesario para recorrer L liñas caché dado un valor de D. Este problema pode ser solventado recorrendo á seguinte fórmula:
    
    \[ R(L,D)=\begin{cases}
	 L,  \text{ se } D\geq8\\ 
	 \ceil[\big]{\frac{8L}{D}},  \text{ se } D<8 
	\end{cases}\]
    
    Polo tanto, o cálculo de R farase dentro do programa C, e este recibirá como parámetros L e D. Con estas premisas, esta e a parte de adquisición de datos do programa:
    
    A continuación, declaramos o array A coa función \_mm\_malloc para aliñar os datos coas liñas caché, e levamos a cabo unha fase de quecemento: escribimos A a certos valores aleatorios acoutados para evitar efectos de inicialización das cachés:
    
    Rematamos a preparación preparando o vector S de resultados e o vector E cos valores dos índices a recorrer en A:
    
    Agora preparamos o experimento propiamente dito: un bucle aniñado realiza a redución mentres se conta o número de ciclos empregados. Unha vez atopados os ciclos medios, imprímense os resultados por saída estándar. Ademais, imprimimos por saída de erro o vector de resultados para asegurar que os cálculos están feitos. A vantaxe de facelo pola saída de erro é que é facilmente desbotable dende bash.
    Por último, liberamos o array A e saímos.
    
    Para automatizar as medicións para diferentes valores de L e D, facemos tamén un script bash sinxelo, que compila o executable cos parámetros de optimización e librerías adecuados, desbota a saída de erro do mesmo e percorre os distintos valores de L e D, repetindo a proba dende fora do executable unha cantidade predefinida de veces. 
    
    O derradeiro compoñente do entorno de probas é un script de R que se encarga da representación gráfica: carga o ficheiro de datos, agrega facendo a mediana as medicións para os mesmos valores de L e D, e debuxa o rendemento fronte a L, agrupando por D en liñas o resultado e mostrando os valores de R correspondentes o carón de cada punto. O eixo de L sigue unha escala logarítmica para apreciar mellor os puntos, xa que os valores de L están concentrados en valores baixos. Isto debe terse en conta á hora de interpretar os resultados.

\section{Resultados}


\subsection{Units}

\subsection{Equations}


\subsection{Some Common Mistakes}

\section{USING THE TEMPLATE}

\subsection{Headings, etc}

Text heads organize the topics on a relational, hierarchical basis. For example, the paper title is the primary text head because all subsequent material relates and elaborates on this one topic. If there are two or more sub-topics, the next level head (uppercase Roman numerals) should be used and, conversely, if there are not at least two sub-topics, then no subheads should be introduced. Styles named ÒHeading 1Ó, ÒHeading 2Ó, ÒHeading 3Ó, and ÒHeading 4Ó are prescribed.

\subsection{Figures and Tables}

Positioning Figures and Tables: Place figures and tables at the top and bottom of columns. Avoid placing them in the middle of columns. Large figures and tables may span across both columns. Figure captions should be below the figures; table heads should appear above the tables. Insert figures and tables after they are cited in the text. Use the abbreviation ÒFig. 1Ó, even at the beginning of a sentence.

\begin{table}[h]
\caption{An Example of a Table}
\label{table_example}
\begin{center}
\begin{tabular}{|c||c|}
\hline
One & Two\\
\hline
Three & Four\\
\hline
\end{tabular}
\end{center}
\end{table}


   \begin{figure}[thpb]
      \centering
      \framebox{\parbox{3in}{We suggest that you use a text box to insert a graphic (which is ideally a 300 dpi TIFF or EPS file, with all fonts embedded) because, in an document, this method is somewhat more stable than directly inserting a picture.
}}
      %\includegraphics[scale=1.0]{figurefile}
      \caption{Inductance of oscillation winding on amorphous
       magnetic core versus DC bias magnetic field}
      \label{figurelabel}
   \end{figure}
   

Figure Labels: Use 8 point Times New Roman for Figure labels. Use words rather than symbols or abbreviations when writing Figure axis labels to avoid confusing the reader. As an example, write the quantity ÒMagnetizationÓ, or ÒMagnetization, MÓ, not just ÒMÓ. If including units in the label, present them within parentheses. Do not label axes only with units. In the example, write ÒMagnetization (A/m)Ó or ÒMagnetization {A[m(1)]}Ó, not just ÒA/mÓ. Do not label axes with a ratio of quantities and units. For example, write ÒTemperature (K)Ó, not ÒTemperature/K.Ó

\section{CONCLUSIONS}

\addtolength{\textheight}{-12cm}   % This command serves to balance the column lengths
                                  % on the last page of the document manually. It shortens
                                  % the textheight of the last page by a suitable amount.
                                  % This command does not take effect until the next page
                                  % so it should come on the page before the last. Make
                                  % sure that you do not shorten the textheight too much.

%%%%%%%%%%%%%%%%%%%%%%%%%%%%%%%%%%%%%%%%%%%%%%%%%%%%%%%%%%%%%%%%%%%%%%%%%%%%%%%%



%%%%%%%%%%%%%%%%%%%%%%%%%%%%%%%%%%%%%%%%%%%%%%%%%%%%%%%%%%%%%%%%%%%%%%%%%%%%%%%%



%%%%%%%%%%%%%%%%%%%%%%%%%%%%%%%%%%%%%%%%%%%%%%%%%%%%%%%%%%%%%%%%%%%%%%%%%%%%%%%%
\section*{APPENDIX}

Appendixes should appear before the acknowledgment.

\section*{ACKNOWLEDGMENT}

The preferred spelling of the word ÒacknowledgmentÓ in America is without an ÒeÓ after the ÒgÓ. Avoid the stilted expression, ÒOne of us (R. B. G.) thanks . . .Ó  Instead, try ÒR. B. G. thanksÓ. Put sponsor acknowledgments in the unnumbered footnote on the first page.



%%%%%%%%%%%%%%%%%%%%%%%%%%%%%%%%%%%%%%%%%%%%%%%%%%%%%%%%%%%%%%%%%%%%%%%%%%%%%%%%

References are important to the reader; therefore, each citation must be complete and correct. If at all possible, references should be commonly available publications.



\begin{thebibliography}{99}

\bibitem{c1} CPU-World. Intel Core i3-3240 specifications. \url{http://www.cpu-world.com/CPUs/Core_i3/Intel-Core\%20i3-3240.html}







\end{thebibliography}




\end{document}
